\chapter*{Abstract}
\addcontentsline{toc}{chapter}{Abstract}

Data-driven models are bringing huge advantages in terms of performance and effectiveness in many sectors.
In the field of Cyber-Physical Systems, although, they are not widespread since their opaqueness and lacks formal proofs pose a serious threat to their employment.

We propose and analyze a novel approach in creating robust and safe controllers drawing from the literature of the adversarial learning.

Our method trains in an adversarial way two Neural Networks to reach a twofold goal:
obtain one network that is able to generate difficult configurations of the environment and another that is able to overcome them in a safe e robust way.
The aim is to create a formally verified controller and, at the same time, to give insights on the most demanding corner cases of a given model.

The approach is promising and worthy of further investigation.


\chapter*{Abstract in italiano}
\addcontentsline{toc}{chapter}{Abstract in italiano}

I modelli data-driven stanno permettendo di raggiungere efficacia e performance mai viste prima in molti settori.
Tuttavia, nel campo dei Cyber-Physical System non sono ancora molto impiegati dal momento che, per via del loro funzionamento difficilmente formalizzabile, non permettono di fare le adeguate verifiche di sicurezza che li renderebbero delle valide alternative ai sistemi attualmente in uso.

Noi proponiamo e analizziamo un nuovo approccio alla creazione di controllori robusti e sicuri attingendo dalla letteratura dell'apprendimento avversario.

Il nostro sistema addestra due Reti Neurali avversarie per ottenere un duplice obbiettivo: ottenere una rete in grado di generare configurazioni insidiose dell'ambiente e un'altra rete in grado di affrontare tali situazioni in modo robusto e sicuro.
Lo scopo è quello di ottenere un controllore verficato formalmente e, allo stesso tempo, ottenere informazioni su quali sono i casi più difficili da gestire per un dato modello.

Il nostro approccio è promettente e sicuramente degno di ulteriori approfondimenti.